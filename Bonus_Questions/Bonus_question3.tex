\documentclass[../../../main.tex]{subfiles}
\graphicspath{../../../Figures/}

\title{Bonus questions}
\author{Fredrik Mårtensson}
\date{\today}

\begin{document}
\maketitle
%\tableofcontents

\section{Bonus questions 4}
\subsection{EDF is not optimal for a task set with precedences. Give an example and introduce an extension to it which is optimal with precedences.} 
As proven by C. L. Liu and J. W. Layland\footnote{\url{http://doi.acm.org/10.1145/321738.321743}} in 1973 the Earliest-deadline-first scheduling\footnote{\url{http://www.cse.chalmers.se/edu/year/2015/course/EDA222_Real_Time_Systems/Documents/Slides/Slides_11.pdf}} is a proven algorithm to handle deadline priorities by dynamically change the order that the system should assign the task with the lowest deadline the highest priority. Advantages with this algorithm is that no manual assignment of priorities is needed and the algorithm can solve this. As the question is described that EDF do not have any priorities should therefor be a false statement. Further considerations are that if the algorithm does not handle priorities correctly, a lower margin for the deadline would be used and thus it will have higher priority by EDF, a long deadline can be described as the system is in no/less hurry to get it executed.
\newline\newline
However, there are problems with EDF when the system becomes overloaded and cannot handle the deadline on all processes, therefore something called Fixed-priority scheduling is used where high priority processes are always executed first and the lower priority processes can miss the deadline. This algorithm is more used than EDF when it comes to industrial real-time systems.

\subsection{Explain briefly the priority inversion and how it is managed.}
Simply said: Priority inversion is a problem, not the solution. 
Let's say we have two processes, one low-priority process and one high-priority process. Let's call them L (low) and H (high), these processes share memory. L starts and enters his critical section. During this time, H starts and because it has higher priority it takes over L. The problem is that L is in critical position and does not leave this one until it is complete. Thus, H has to wait despite its priority and is thus locked until L is finished. The word inversion does in this case mean that a high priority process is put at the bottom as the low priority process prevents it from finishing.
\newpage
\subsubsection{Simplified scenario for better explaination}
Lets say we have three processes, L, M, H where each stands for: Low, medium and high priority. These three processes share the same critical memory.
\begin{enumerate}
    \item L is running, M and H is not.
    \item L enter its critical section.
    \item H starts and now have the highest priority.
    \item H tries to access the critical section, Waiting for L to finish.
    \item M starts and now have higher priority than L and therefor continue and L is paused.
\end{enumerate}
\textbf{So what happened?}
\begin{enumerate}
    \item  H is placed in the bottom since M is running
    \item  L wont start until M is finished
    \item H cant start since L isn't finished
\end{enumerate}
\textbf{Solution:} In order to handle this L priority must be raised (Priority inheritance) so M isn't prioritised, until L leaves the critical section so H can continue. Another solution is randomly raise the priority on processes that is holding locks (Random boosting).
\newline\newline
For some fun reading about Priority inversion and why it can cause huge problems \footnote{\url{https://users.cs.duke.edu/~carla/mars.html}}



\end{document}