\documentclass[../../../main.tex]{subfiles}
\graphicspath{../../../Figures/}

\title{Bonus questions}
\author{Fredrik Mårtensson}
\date{\today}

\begin{document}
\maketitle
%\tableofcontents

\section{Bonus questions 2}
\subsection{Explain briefly the Peterson’s algorithm and describe how it achieves mutual exclusion.}
Peterson's algorithm prevents a deadlock by telling a process that uses the same function or part of a memory that it is occupied and used by another process. This means that only one process can use the function/memory. It uses a bool array flag (status) with the size two and an int variable called turn (which task should have priorit). When process start the flags is set to false, when an execution of critical section the flag[0] is set to true. If the other process does not want to enter its critical section (Flag[1] = False) or if turn is set to 0 (giving the first process priority to enter before the second process is going into the critical section) then the first process may enter its critical section. The flag will be set to false when execution is completed. Peterson's algorithm will ensure that two process will not go into their critical sections at the same time using the flag and turn variable. According to wikipedia\footnote{\url{https://en.wikipedia.org/wiki/Peterson\%27s_algorithm}} the following rules apply where P0 is first process and P1 is the second process:

\begin{table}[H]
\begin{tabular}{|l|l|l|l|}
\hline
Flag{[}0{]} & Flag{[}1{]} & Turn & Critical \\ \hline
True & True & 0 & P0 \\ \hline
True & True & 1 & P1 \\ \hline
True & False & 0 & P0 \\ \hline
True & False & 1 & P0 \\ \hline
False & True & 0 & P1 \\ \hline
False & True & 1 & P1 \\ \hline
\end{tabular}
\end{table}

Additional great site is Introduction to Deadlocks\footnote{\url{https://cs.nyu.edu/courses/spring03/V22.0202-002/lecture-06.html}} and Solutions to the Critical Section Problem\footnote{\url{http://www2.cs.uregina.ca/~hamilton/courses/330/notes/synchro/node3.html}}


\section{Bonus questions 3}
\subsection{Explain briefly (at least three) existing techniques to avoid deadlock in multi-threaded programs}
According to the Coffman conditions that include: 
\begin{enumerate}
    \item \textbf{Mutual exclusion}, mean that if a resource is shared (shareable) with two or more processes. As an example it could be if two processes try to change the same part of the memory or utilising the same function and could therefor overwrite the data from one of the processes that would in turn could create a deadlock. By using methods such as Peterson’s algorithm that allows only one function to access a critical section at the time would solve this scenario.
    
    \item \textbf{Hold and wait}, means that if one process is holding a resource and waiting for a resource from another process then a deadlock could occur. To solve this an allocation from the beginning of the resources may be needed. This however is not performance efficient as the resources may be poorly used. As given from geeksforgeeks\footnote{\url{https://www.geeksforgeeks.org/deadlock-prevention/}} a possibility could be that we allocate resources from a printer and then hold them, this means that the printer will be blocked from other processes until the execution is finished.
    \item \textbf{No preemption}, The process should have to release its resources and not be taken away (forced). This is however hard to achieve as a TTL exist and if the process use more time than it is given then it might be terminated.
    \item \textbf{Circular wait}, if process one is waiting for process two and process two is waiting for process one an deadlock will occur. This mean that it must be a chain of processes, each member of the chain is waiting for resources that is held by the next member of the specified chain. All resources should me numbered and the process should always work in a chain without moving to next process until the previous is finished. 
\end{enumerate}

There are multiple algorithms that is used to prevent deadlock, as example there is:
\begin{enumerate}
    \item Peterson’s algorithm, Mentioned in bonus question two (added in same file)
    \item Banker's algorithm
    \item Preventing recursive locks
    
\end{enumerate}
\end{document}